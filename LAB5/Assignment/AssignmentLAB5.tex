\documentclass[unicode,11pt,a4paper,oneside,numbers=endperiod,openany]{scrartcl}

\usepackage{assignment}
\usepackage{textcomp}
\usepackage{url}
\hyphenation{PageRank}
\hyphenation{PageRanks}

\definecolor{Brown}{cmyk}{0,0.81,1,0.60}
\definecolor{Green}{rgb}{0.516, 0.5976, 0.480}
\definecolor{OliveGreen}{cmyk}{0.64,0,0.95,0.40}
\definecolor{CadetBlue}{rgb}{0.17,0.30,0.547}
\definecolor{Red}{rgb}{0.684,0.156,0.32}
\lstset{
  language=matlab,
  basicstyle=\ttfamily,
  frame=ltrb,
  framesep=5pt,
  basicstyle=\scriptsize,
  stringstyle=\ttfamily\color{Brown}\bfseries,
  commentstyle=\ttfamily\color{Green}\bfseries,
  keywordstyle=\ttfamily\color{CadetBlue}\bfseries,
  identifierstyle=\ttfamily, 
  tabsize=2,
  showstringspaces=true,
  numbers=none
}



\begin{document}

\setassignment
\setduedate{Thursday 22 October 2015, 10:30}
\serieheader{Introduction to PDEs (LAB)}{Academic Year 2015/2016}{Instructor: Prof. Rolf Krause}{Dr. Drosos Kourounis}{Assignment 5 - Finite Element Solution of Poisson's equation in 3D}{}

\section*{Solution of Poisson's equation}
We seek the discrete solution of Poisson's equation
\begin{align*}
-&\nabla^2 u(x, y, z) = f(x,y, z), \quad \in \Omega \\
&\frac{\partial{u}}{\partial n}       = 0, \quad y=1, z=1,\\
&u(x, y, z) = u_0(x, y, z), \text{otherwise},
\end{align*}
where $\Omega = [0,1]^3$. We discretize the domain $\Omega$ using a quadrilateral $N_x \times N_y \times N_z$ grid of trilinear elements. 
\begin{enumerate}
\item Find the analytical expression of $f(x,y)$ so that the exact 
solution of the PDE is \[u_0(x,y) =x\, e^{-(y-1)^2 (z-1)^2}.\] 
\item Verify that the exact solution satisfies the homogeneous Neumann conditions
at $y=1$ and at $z=1$. 
\item Implement the solution in \small{MATLAB} following the steps described below.
\end{enumerate}


\section{Mesh generation}
Modify the mesh generation routine you already have,
so that it assumes three input arguments for the cube dimensions $L_x,L_y,L_z$ and 
three for the grid size $N_x,N_y,N_z$. It outputs the hexahedral or tetrahedral mesh of the cube.


\section{Finite element assembly}
\begin{itemize}
\item Compute by hand the mass matrix and the Laplacian on the Hexahedron $[0, 1]^3$ and
the Tetrahedron [(0,0,0), (1, 0, 0), (0, 1, 0), (0, 0, 1)].
\item Compute the mass and Laplacian matrices on the single-element grid consisting
of the hexahedron or the tetrahedron you used in the previous step using your code. 
\end{itemize}


\section{Local operators}
Modify the following functions to obtain the local operators for the hexahedral.
\begin{center}
$\displaystyle M_e = \int_{V_e} N_i N_j d V_e$
\end{center}
\lstinputlisting[language=matlab,firstnumber=1,basicstyle=\ttfamily\scriptsize]{Msymbolic.m}
\begin{center}
$\displaystyle K_e = \int_{V_e} \nabla N_i \cdot \nabla N_j d V_e$
\end{center}
\lstinputlisting[language=matlab,firstnumber=1,basicstyle=\ttfamily\scriptsize]{Ksymbolic.m}

\section{Enforce boundary conditions}
Describe the necessary modifications to the code so that you enforce 
Dirichlet conditions such that the symmetry of the matrix is maintained.



\section{Linear system}
\begin{enumerate}
\item
Use the direct sparse solver implemented in \small{MATLAB} (backslash)
to solve the linear system for $h = \delta z = \delta y = \delta x = 1/N$ for $N=10,20,40,80,160$.
Create a table listing the running time of the direct sparse solver.
%\item 
%For $N$ as above solve the linear system using the following iterative methods
%\begin{enumerate}
%\item PCG
%\item MinRes
%\item SymmLQ
%\item LSMR
%\item CGS
%\item BiCGStab
%\end{enumerate}
%without preconditioner, with diagonal preconditioner  and with incomplete Cholesky 
%preconditioner, provided in \small{MATLAB} by the command ichol. For each case
%set $tol = 10^{-8}$ and $maxiters = 1000$. For each preconditioner plot the relative
%residual (use logarithmic scale for the the y-axis) at each iteration (x-axis). Use a single
%plot for all iterative methods. You should present 3 plots, one for each preconditioner.
%\item
%Additionally provide a table listing for each preconditioner the number of iterations 
%the time spent until convergence and the final residual. 
\end{enumerate}

\section{Convergence study}
Perform a study of the convergence. Similarly to
previous assignments compute the $L^2$ and $H^1$ norms of the error and plot it {\bf using logarithmic scale on both axes}
as a function of $h = \delta z = \delta y = \delta x = 1/N$ for $N=10,20,40,80,160$.
Note that the discrete $L^2$ and $H^1$ norms are easilly obtained from
\begin{align*}
\|u - u_h\|_{L^2} &= \sqrt{(u-u_h)^T M (u-u_h)} \\
\|u - u_h\|_{H^1} &= \sqrt{(u-u_h)^T M (u-u_h) + (u-u_h)^T K (u-u_h)},
\end{align*}
where $u_h$ is the vector of the discrete finite element solution you 
obtained by solving the linear system $K u_h = b$ and $u$ is the vector
of the exact values of the solution (recall that the exact solution
is given by $u_0(x,y,z)$) evaluated at the grid-points. 

\section{Performance study}
For each $N$ of the previous question, compute the time needed for the assembly,
and the solution of the linear system and plot them on the same plot on logarithmic 
scale on both axes.

\section{Visualization}
For both grid types visualize the solution using (ViSiT) for three values of $N$, $N=10,80,160$ 
and also visualize the exact solution at the end (as the fourth plot) for comparison.

\end{document}
