\documentclass[unicode,11pt,a4paper,oneside,numbers=endperiod,openany]{scrartcl}

\usepackage{assignment}
\usepackage{textcomp}
\usepackage{url}
\hyphenation{PageRank}
\hyphenation{PageRanks}

\definecolor{Brown}{cmyk}{0,0.81,1,0.60}
\definecolor{Green}{rgb}{0.516, 0.5976, 0.480}
\definecolor{OliveGreen}{cmyk}{0.64,0,0.95,0.40}
\definecolor{CadetBlue}{rgb}{0.17,0.30,0.547}
\definecolor{Red}{rgb}{0.684,0.156,0.32}
\lstset{
  language=matlab,
  basicstyle=\ttfamily,
  frame=ltrb,
  framesep=5pt,
  basicstyle=\scriptsize,
  stringstyle=\ttfamily\color{Brown}\bfseries,
  commentstyle=\ttfamily\color{Green}\bfseries,
  keywordstyle=\ttfamily\color{CadetBlue}\bfseries,
  identifierstyle=\ttfamily, 
  tabsize=2,
  showstringspaces=true,
  numbers=none
}



\begin{document}

\setassignment
\setduedate{Thursday 8 October 2015, 10:30}
\serieheader{Introduction to PDEs (LAB)}{Academic Year 2015/2016}{Instructor: Prof. Rolf Krause}{TA: Drosos Kourounis}{Assignment 3 - Finite Element Solution of Poisson's equation in 2D}{}


\section*{Solution of Poisson's equation}
We seek the discrete solution of Poisson's equation
\begin{align*}
-\nabla^2 u(x, y) &= f(x,y), \quad \in \Omega \\
\frac{\partial{u}}{\partial n}       &= 0, \quad (x, y) \in \partial \Omega,
\end{align*}
where $\Omega = [0,1]^2$.
\begin{enumerate}
\item Derive the analytical expression of $f(x,y)$ so that the exact 
solution of the PDE is \[u_0(x,y) =x^2\, y^2\, \left(2\, x - 3\right)\, \left(2\, y - 3\right).\] 
\item Verify that the exact solution satisfies the homogeneous Neumann conditions
at the boundary $\partial \Omega$. 
\item Write \small{MATLAB} code that solves the PDE 
implementing the steps following\footnote{ 
You should include in your submission only the implementations of the
functions marked with red and not the whole \small{MATLAB} code.}
\end{enumerate}

\section{Mesh generation}
Write a \small{MATLAB} function that generates a $N_x \times N_y$ quadrilateral grid.
The input arguments of the function should be \lstinline+N_x, N_y, grid_type+. If third argument \lstinline+grid_type+
is equal to \lstinline+'triangles'+ then you can easilly generate a triangular grid by dividing each quadrilateral
to 2 triangles drawing the diagonal connecting the local points 1 and 3. 
The output of the function should be the matrix of the connectivity of the grid 
(\lstinline+Elements+) and the matrix of the coordinates of the grid points (\lstinline+Points+). 
The matrix \lstinline+Elements+ should by of size $N_e \times N_\text{v}$,
where the $N_e$ number of elements and $N_\text{v}$ the number of vertices. The matrix
\lstinline+Points+, should be of size $N \times d$, where $N$ the number of points and $d$ the number of 
space dimensions.
\begin{lstlisting}[numbers=left, numberstyle=\tiny, stepnumber=1, numbersep=10pt, identifierstyle=\ttfamily\color{Red}\bfseries]
function [mesh] = makeGrid(L_x, L_y, N_x, N_y, grid_type)
\end{lstlisting}
The output \lstinline+mesh+ is a MATLAB structure that contains all the information that are needed for the mesh generation
and mesh description, namely, \lstinline+delta_x, delta_y, N_x, N_y, L_x, L_y, N_e, N_v, N, Elements, Points, PointMarkers+. 
The array \lstinline+PointMarkers+ should be zero at every point except those points that belong to the boundary. We will 
need it later to enforce Dirichlet boundary conditions to our Discrete system of equations.
Use the MATLAB function \lstinline+writeMeshAsVTKFile.m+ in the \lstinline+matlab+ folder
and call it passing the structure mesh and the name of the output file.
This will dump the grid as a VTK file that you can visualize using visualization package ViSiT. The declarations follow.
\begin{lstlisting}[numbers=left, numberstyle=\tiny, stepnumber=1, numbersep=10pt]
function writeMeshAsVTKFile(mesh, vtkfile)
\end{lstlisting}



\section{Finite element assembly}
Use the following code sample for the assembly of your discrete operators.
\begin{lstlisting}[numbers=left, numberstyle=\tiny, stepnumber=1, numbersep=10pt]
function [M,K,b] = assembleDiscreteOperators(mesh)
  N   = mesh.N;
  Ne  = mesh.Ne;
  Nv  = mesh.Nv;
  M   = zeros(N,N); K = zeros(N,N);
  b   = zeros(N,1);
  for e=1:Ne
    Me = makeMe(e, mesh);
    Ke = makeKe(e, mesh);
    fp = makebe(e, mesh);
    for i=1:Nv
      I = mesh.Elements(e, i);
      for j=1:Nv
        J = mesh.Elements(e, j);
        M(I, J) = M(I, J) + Me(i, j);
        K(I, J) = K(I, J) + Ke(i, j);
      end % j loop
      be   = Me*fp;
      b(I) = b(I) + be(i);
    end % i loop
  end % e loop
end
\end{lstlisting}
Provide implementations of the individual 
functions needed for forming the element mass matrix \lstinline+M_e+, element Laplacian \lstinline+K_e+ 
and the element rhs \lstinline+b_e+. 
\begin{lstlisting}[numbers=left, numberstyle=\tiny, stepnumber=1, numbersep=10pt, identifierstyle=\ttfamily\color{Red}\bfseries] 
function Me = makeMe(e, mesh);
function Ke = makeKe(e, mesh);
function fp = makebe(e, mesh);
\end{lstlisting}
Keep in mind that \lstinline+f_p+ should hold the values of the function $f(x,y)$
evaluated at the nodes of the $e$th element,  for example
$f_p = \left [ f(x^e_1, y^e_1) \; f(x^e_2, y^e_2) \; f(x^e_3, y^e_3)  \right ]^T$.


\section{Solution}
Provide a function that solves the linear system $K u_h = b$, and use it to obtain the solution of the linear system
you constructed in the previous step. Can the system be solved? Any ideas why? Think of a way to enforce in the discrete 
system $K u_h = b$ the Dirichlet boundary condition $u_h(1) = 0$ at the first point of the grid, the point (0,0). 
Is this condition consistent with the exact solution?
If yes why? 
\begin{lstlisting}[numbers=left, numberstyle=\tiny, stepnumber=1, numbersep=10pt, identifierstyle=\ttfamily\color{Red}\bfseries] 
function x = solve(K, b)
\end{lstlisting}

\section{Convergence study}
Compare the solution you obtained with the exact one in the first part $u_0(x,y)$ by computing the 
$L^2$ and $H^1$ norms of the error and plot them using logarithmic scale on both axes
as a function of $h = \max{(\delta x, \delta y)}$ for $N_x = N_y = 5,10,20,40$.
Note that the discrete $L^2$ and $H^1$ norms are easilly obtained from
\begin{align*}
\|u - u_h\|_{L^2} &= \sqrt{(u-u_h)^T M (u-u_h)} \\
\|u - u_h\|_{H^1} &= \sqrt{(u-u_h)^T M (u-u_h) + (u-u_h)^T K (u-u_h)},
\end{align*}
where $u_h$ is the vector of the discrete finite element solution you 
obtained by solving the linear system $K u_h = b$ and $u$ is the vector
of the exact values of the solution. The exact solution
is given by $u_0(x,y)$ evaluated at the grid points.

%\section{Visualization}
%For both grid types visualize the solution using (ViSIT) for three values of $N$, $N=10,320,1280$ 
%and also visualize the exact solution at the end (as the fourth plot) for comparison.


\end{document}
