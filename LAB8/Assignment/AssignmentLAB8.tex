\documentclass[unicode,11pt,a4paper,oneside,numbers=endperiod,openany]{scrartcl}

\usepackage{assignment}
\usepackage{textcomp}
\usepackage{url}
\hyphenation{PageRank}
\hyphenation{PageRanks}

\definecolor{Brown}{cmyk}{0,0.81,1,0.60}
\definecolor{Green}{rgb}{0.516, 0.5976, 0.480}
\definecolor{OliveGreen}{cmyk}{0.64,0,0.95,0.40}
\definecolor{CadetBlue}{rgb}{0.17,0.30,0.547}
\definecolor{Red}{rgb}{0.684,0.156,0.32}
\lstset{
  language=matlab,
  basicstyle=\ttfamily,
  frame=ltrb,
  framesep=5pt,
  basicstyle=\scriptsize,
  stringstyle=\ttfamily\color{Brown}\bfseries,
  commentstyle=\ttfamily\color{Green}\bfseries,
  keywordstyle=\ttfamily\color{CadetBlue}\bfseries,
  identifierstyle=\ttfamily, 
  tabsize=2,
  showstringspaces=true,
  numbers=none
}

\def\u{{\bf u}}
\def\c{{\bf c}}
\def\f{{\bf f}}
\def\n{{\bf n}}
\def\d{{\bf d}}
\def\w{{\bf w}}
\def\v{{\bf v}}
\def\e{{\bf e}}
\def\b{{\bf b}}
\def\x{{\bf x}}
\def\s{{\bf s}}
\def\g{{\bf{G}}}
\def\p{{\bf{g}}}
\def\h{{\bf h}}
\def\A{{\bf A}}
\def\H{{\bf H}}
\def\Z{{\bf Z}}
\def\W{{\bf W}}
\def\0{{\bf 0}}

\begin{document}

\setassignment
\setduedate{Thursday 29 October 2015, 10:30}
\serieheader{Introduction to PDEs (LAB)}{Academic Year 2015/2016}{Instructor: Prof. Rolf Krause}{Dr. Drosos Kourounis}{Assignment 6 - Finite Element Solution of vector Poisson's equation in 3D}{}

We seek the discrete solution of the vector Poisson's equation
\begin{align*}
-&\nabla^2 \u(x, y, z) = \f(x,y, z), \quad \in \Omega \\
&\n \cdot \nabla \u   = 0, \quad y=1, z=1,\\
&\u(x, y, z) = \u_0(x, y, z), \text{otherwise},
\end{align*}
where $\Omega = [0,1]^3$. We discretize the domain $\Omega$ using a quadrilateral $N_x \times N_y \times N_z$ grid of trilinear elements. 
\begin{enumerate}
\item Find the analytical expression of $\f(x,y)$ so that the exact 
solution of the PDE is \[u_{0,x} = u_{0,y} = u_{0,z} =x\, e^{-(y-1)^2 (z-1)^2}.\] 
Note that $\f$ is a vector. Please explain what are the values of each comment.
\item Solve the problem by assuming an ordering of the variables per vector component,
more precisely assume that the solution vector is 
\begin{align}
\left (
\begin{array}{ccc}
U_x^T & U_y^T & U_z^T
\end{array}
\right )^T
\end{align}
where $U_x$ contains all the unknowns corresponding to the $x$ component, etc.
\begin{itemize}
\item Solve the related linear system using the backslash operator and the CG
algorithm with Cholesky preconditioning. Use the object-oriented MATLAB code that was given to you.
\item Write on a small table the timings using numbers with one decimal digit only.
\item Plot the solution using visit and give the norms of the errors for $N=10, 20, 40$.
\end{itemize}
\item Next reorder the variables so that each node contains all the components. 
Your solution vector obtains the form:
\begin{align}
\left (
\begin{array}{ccccccc}
u_{1_x} & u_{1_y} & u_{1_z} & \cdots & u_{n_x} & u_{n_y} & u_{n_z}
\end{array}
\right )^T
\end{align}
For the permutation (reordering) of the original matrix and vector familiarize your self
with the MATLAB function \lstinline+repmat+\footnote{Experiment with a 6 x 6 matrix to understand permuting a matrix and permuting it back. If the permutation routine works correctly then permuting the matrix back using the inverse permutation should result to the original matrix before the permutation was applied. Suppose $p$ is the permutation vector. Then the inverse permutation $pinv$ satisfies $p(pinv(i)) = pinv(p(i)) = i$}. Compute the permutation array
and then write a function that accepts the matrix to be permuted \lstinline+A+,
the vector (rhs) to be permuted and the permutation array \lstinline+p+,
    and returns the permuted matrix and vector. 
\begin{itemize}
\item Use \lstinline+spy()+ to see the result of the permutation. And provide in 
your report pictures of both the original matrix and the permuted one. 
\item Solve the permuted system measuring the times for the backslash
and CG with Cholesky preconditioning. Note that for using Cholesky, you should apply the 
boundary conditions maintaining the symmetry of the matrix. 
Complete the table you created previously with the running times of backslash
and CG for the permuted system. How do they compate with those of the original
unpermuted system?
\item Permute back the solution to the original ordering and compare 
to the solution you obtained at the previous step. Are they the same?
What are the error norms?
\end{itemize}
\item[Hint:] {\it Suppose that you have a 6x6 matrix A. The matrix A is identical with 
the permuted matrix A(p,p) where p is the permutation array \lstinline+p=[1 2 3 4 5 6]+. 
So yes, in order to permute a matrix in MATLAB you only need to write \lstinline+B=A(p,p)+,
where \lstinline+p+ is the permutation array. How do you permute a vector? 
In order ta familiarize yourself with all of that you can use the symbolic package.
The command \lstinline+A=sym('A', [6,6])+ will create a symbolic matrix 6 by 6.
You can permute it in any way you like by creating a permutation array p. For 
example to permute the 3 first rows and columns to the 4th,5th and 6th correspondingly,
you just need to introduce a permutation array \lstinline+p=[4 5 6 1 2 3]; B=A(p,p)+.}
\item Finally assemble the matrix from the very beginning so that its ordering
is identical to that of the previous step, meaning that
\begin{itemize}
\item you need to assemble vector mass and stiffness element matrices, 
\item you need to provide the spy plot of the local vector stiffness matrix and the global stiffness matrix,
\item you need to compare the solution with the solution of the previous step. 
\item attach the code for the vector assembly, the code for generating the permutations 
of the matrix and the vectors and the inverse permutation array. 
\end{itemize}

\end{enumerate}
For this assignment you should use the object-oriented code that was provided to you. Think how the code should be structured introducing new methods in the class that exploit the existing methdos where needed and introduce entirely new methods for the vector assembly. 




\end{document}
